%----------------------------------------------------------------
\usepackage{verbatim}
\usepackage{algorithm}
\usepackage{algpseudocode}
\usepackage{pifont}
\usepackage{amsmath,amsthm,amssymb,amsfonts,multicol}
\usepackage{makeidx}
\usepackage{graphicx}
\usepackage{xcolor}
\usepackage{float}
\usepackage{booktabs}
\usepackage[a4paper,vmargin=3.5cm,left=3cm,right=4cm]{geometry}
\usepackage[colorlinks=true,
	linkcolor=blue!30!black]
	{hyperref}
\usepackage{glossaries}
\usepackage{thesis-utils}
\usepackage[nottoc]{tocbibind}
%----------------------------------------------------------------

%----------------------------------------------------------------
\usepackage[localise=on]{xepersian}
\usepackage{xepersian}
\settextfont[Scale=1.1]{XB Roya}
\setdigitfont[Scale=1]{XB Roya}
%\setlatintextfont[Scale=1]{Tahoma}
%\defpersianfont\Nastaliq[Scale=1.5]{IranNastaliq}
%\defpersianfont\Titre[Scale=1]{XB Titre}

\def\d{\displaystyle}
\def\term{\mathrm{term}}
\allowdisplaybreaks 
%=================================================
\theoremstyle{plain} 
\newtheorem{theorem}{قضیه}[chapter]
\newtheorem{lemma}{لم}[chapter]
\newtheorem{proposition}{گزاره}[chapter]
\newtheorem{corollary}{نتیجه}[chapter]
\newtheorem*{conjecture}{حدس}
\theoremstyle{definition}
\newtheorem{definition}{تعریف}[chapter]
\newtheorem{example}{مثال}[chapter]
\newtheorem{prob}{مساله}[chapter]
\theoremstyle{remark} 
\newtheorem{remark}{ملاحظه}[chapter]
%==================================================
\setcounter{secnumdepth}{3}
%\setcounter{chapter}{4}
\renewcommand{\bibname}{مراجع}
\oddsidemargin =0cm 
\evensidemargin =0cm
\textheight =21cm
\textwidth= 15.3cm
\headsep= 1.2cm
\topmargin  =0cm
\usepackage{perpage}
%\setcounter{chapter}{3}
\MakePerPage{footnote}


% USEFUL COMMANDS
\newcommand{\blankpage}{\thispagestyle{empty}
\begin{center}%
\textit{این صفحه تعمداً خالی گذاشته شده است}%
\end{center}
\clearpage
}

\baselineskip = 1.15cm
\renewcommand{\baselinestretch}{2.3}

