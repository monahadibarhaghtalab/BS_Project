{%Open the group
	\makeCondensedChap
	\chapter*{چکیده}
}

شماهای رمزنگاری مبتنی بر نظریه اعداد، همانند  $RSA$ و رمزنگاری بر مبنای خم‌های بیضوی، در صورت تحقق کامپیوترهای کوانتمی، در ابعاد بزرگ می‌شکنند. توسط متخصصان حوزه رمزنگاری ادعا شده‌است، رمزنگاری مشبکه مبنا  یک روش پسا کوانتومی است و در مقابل کامپیوترهای کوانتومی نمی شکند. در رمزنگاری امن-اثبات‌پذیر، امنیت یک شمای رمزنگاری با فرض سخت‌ بودن یک مسئله‌ی ریاضی مشهور اثبات می‌گردد. در رمزنگاری مشبکه مینا، شماهای امن-اثبات پذیر از توزیع گوسی گسسته نمونه‌برداری می‌شود. یکی از عملیات‌های زمان‌بر در شماهای رمزنگاری مشبکه مبنا  نمونه‌برداری  گسسته از  این توزیع است. برای نمونه‌برداری از این توزیع روش‌های متفاوتی وجود دارد. هر کدام از این روش‌ها سرعت و میزان حافظه مصرفی متفاوتی دارند. با توجه به تمایل ما به بهینه بودن این نمونه‌برداری، که باعث بهبود رمزنگاری می‌شود، تلاش است تا سرعت را بالا ببریم و استفاده از حافظه را پایین بیاوریم. در جریان این مصالحه روش  آلیاس در مقابل روش تجمعی - که سریعترین روش استفاده شده در کارهای قبلی مرتبط می‌باشد - از نظر حافظه در یک مرتبه و از نظر سرعت در مرتبه بالاتری قرار دارد . در این مقاله سعی شده است بعد از توضیح کوتاهی در رابطه با مفاهیم اولیه، روش‌های مختلف نمونه‌برداری را توضیح داده و مقایسه‌ای انجام گیرد. در انتها روش آلیاس مورد بررسی قرار می گیرد و نتایج به‌دست‌آمده در مقایسه با روش تجمعی نشان داده می‌شود. 

کلید واژه : رمزنگاری، مشبکه مبنا، توزیع گوسی، توزیع گسسته، تجمعی، آلیاس