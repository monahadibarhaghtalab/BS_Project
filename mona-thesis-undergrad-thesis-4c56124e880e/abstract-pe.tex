\chapter*{چکیده}
رمزنگاری مشبکه مبنا به‌دلیل سختی زیادی که دارد ، برای کاربردهای متنوعی به کار می‌رود. در این رمزنگاری، برای تولید اعداد تصادفی از توزیع گسسته گوسی نمونه‌برداری می‌شود. برای نمونه‌برداری از این توزیع روش‌های متفاوتی وجود دارد . هر کدام از این روش‌ها سرعت و میزان حافظه مصرفی متفاوتی دارند. با توجه به تمایل ما به بهینه بودن این نمونه‌برداری که باعث بهبود رمزنگاری می‌شود ، تلاش است تا سرعت را بالا ببریم و استفاده از حافظه را پایین بیاوریم. در جریان این مصالحه روش $alias$ در مقابل روش تجمعی - که سریعترین روش موجود است- از نظر حافظه در یک مرتبه و از نظر سرعت در مرتبه بالاتری قرار دارد . در این مقاله سعی شده است بعد از توضیح کوتاهی در رابطه با مفاهیم اولیه ، روش‌های مختلف نمونه‌برداری را توضیح داده و مقایسه‌ای انجام گیرد . در انتها روش $alias$ مورد بررسی قرار می گیرد و نتایج به دست آمده در مقایسه با روش تجمعی نشان داده می‌شود. 
\\
کلید واژه : رمزنگاری ، مشبکه مبنا ، نمونه‌برداری، توزیع گوسی، توزیع گسسته، تجمعی